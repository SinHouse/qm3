\normalsize
\subsection[\_shm]{\_shm.so}
This binary module (must be compiled) provides basic SHM functionality within python.\\
The shared memory can be reserved via python script (\func{alloc}) or directly accessed once known the allocation identifier (\parm{shmid}).
The main purpose of this module is to speed-up the communication layer with third party programs (engines), removing the need for
physical files (SHM works directly on RAM) to communicate with the external engines (ie, updating partial charges, coordinates, gradients, ...).\\
It provides 2 basic mechanisms: \parm{read}/\parm{write} in which binary strings (bytes) are managed (and parsed/generated using the struct.pack/unpack mechanism), or direct acces to lists of floats (\parm{read\_r8}/\parm{write\_r8}).
\begin{pyglist}[language=python,fvset={frame=single}]
shmid = _shm.alloc( size_in_bytes )

_shm.clean( shmid )

_shm.write( shmid, "bytes" )

"" = _shm.read( shmid, size_in_bytes )

_shm.write_r8( shmid, [ float ] )

[ float ] = _shm.read_r8( shmid, num_of_floats )
\end{pyglist}

\footnotesize
\begin{pyglist}[language=python,fvset={frame=single}]
import qm3.utils._shm
import struct

n = 10
p = qm3.utils._shm.alloc( n * 8 )

qm3.utils._shm.write( p, b"".join( [ struct.pack( "d", i ) for i in range( n ) ] ) )
print( qm3.utils._shm.read_r8( p, n ) )

qm3.utils._shm.write_r8( p, [ float( i ) for i in range( n ) ] )
print( list( struct.unpack( "%dd"%( n ), qm3.utils._shm.read( p, n * 8 ) ) ) )

qm3.utils._shm.clean( p )
\end{pyglist}

%\begin{pyglist}[fvset={frame=single}]
%#SOURCE@../samples/logs/
%\end{pyglist}
