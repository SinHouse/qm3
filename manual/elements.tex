\normalsize
\subsection[elements]{elements.py}
This module offers information about the chemical elements of the periodic table:

\begin{pyglist}[language=python,fvset={frame=single}]
mass     # g/mol
r_vdw    # Ang
r_cov    # Ang
r_sol    # Ang
ionpot   # eV / Filimovov et al. [10.1080/10629360903438370]
eafin    # eV / Filimovov et al. [10.1080/10629360903438370] / Myers [10.1021/ed067p307]
symbol
rsymbol

def calc_mass( formula = "H2O1" )
\end{pyglist}
the information is coded using dictionaries for each variable: 
\begin{itemize}
\item atomic masses (\func{mass} in g/mol)
\item van der Waals radiis (\func{r\_vdw} in Å)
\item covalent radiis (\func{r\_cov} in Å)
\item solvation radiis (\func{r\_sol} in Å)
\item ionization potentials (\func{ionpot} in eV)
\item electro-affinites (\func{eafin} in eV)
\item atomic symbols (\func{symbol})
\item atomic numbers (\func{rsymbol})
\end{itemize}
All, except \func{rsymbol}, provide the custom information using as index the atomic number; being \func{rsymbol} the opposite case.
Finally, there is a function which provides basic calculation of molecular masses (\func{calc\_mass}), for which is mandatory
to especify the number of atoms in the molecule (even if there is only one of this kind).

\footnotesize
\begin{pyglist}[language=python,fvset={frame=single}]
>>> import qm3.elements
>>> qm3.elements.mass[16]
32.065
>>> qm3.elements.symbol[16]
'S'
>>> qm3.elements.rsymbol["S"]
16
>>> qm3.elements.calc_mass( "H2S1O4" )
98.07848
\end{pyglist}
